\section{Trabalhos relacionados}
\label{sec:related}

Operadores podem utilizar mensagens dos protocolos de roteamento (e.g.,
OSPF e IS-IS) e arquivos de configuração dos roteadores para mapear a
topologia de sua rede~\cite{feamster05nsdi, turner10cenic,
athina08link}.  Esta abordagem resulta em mapas completos e precisos da
topologia, mas só está disponível para operadores de redes e é restrita
a uma única rede.  Para mapear múltiplas redes, podemos utilizar
coletores públicos de mensagens BGP\footnotemark{} para construir um
mapa dos sistemas autônomos da Internet~\cite{oliveira08as2tier,
dhamdhere08evo}.  Infelizmente, o BGP não expõe todos os enlaces da rede
e coletores públicos de mensagens BGP não cobrem todos os sistemas
autônomos da Internet~\cite{oliveira08as2tier, cohen06darkmatter}.
Neste trabalho tomamos a abordagem ortogonal de medir a topologia da
rede no nível de roteadores usando medições ativas.

\footnotetext{The University of Oregon Routeviews Project,
http://www.routeviews.org\newline{}
\indent{}\indent{}\hspace{-1mm}RIPE Routing Information Service,
http://www.ripe.net/data-tools/stats/ris}

Pesquisa sobre mapeamento topológico no nível de roteadores usando
medições ativas têm três objetivos principais: (i) aumentar a cobertura
da Internet, (ii) aumentar a precisão da topologia construída e (iii)
aumentar a frequência das medições.

A abordagem clássica para aumentar a cobertura da Internet é monitorar
um grande número de caminhos.  A plataforma Skitter/Ark da
CAIDA~\cite{skitter} tenta cobrir toda a Internet usando alguns
monitores para medir caminhos para todos os prefixos $/24$ anunciados na
Internet.  O problema é que o Skitter/Ark demora de dois a três dias
para coletar a topologia devido ao grande número de caminhos monitorados
e limitações de banda nos monitores.  Uma alternativa é dividir a carga
de sondagem das medições entre vários monitores, como nos sistemas
DIMES~\cite{shavitt09dimes} e Ono~\cite{choffnes10crowd}.  Neste
trabalho assumimos que o conjunto de monitores e destinos é fixo.
Porém, utilização do RemapRoute é ortogonal ao conjunto de monitores e
destinos.  O RemapRoute pode ser usado por qualquer um dos sistemas
acima para reduzir a sobrecarga de rede.

Técnicas para aumentar a precisão da topologia coletada tentam inferir
mais informações sobre a rede do que medições tradicionais com
traceroute.  O Paris traceroute envia sondas adicionais variando
sistematicamente os valores nos campos do cabeçalho IP para detectar
todos os roteadores que fazem balanceamento de carga em uma
caminho~\cite{augustin07, veitch09balancer}.  O \rmprt{} detecta
roteadores que fazem balanceamento de carga utilizando o mesmo algoritmo
que o Paris traceroute.  O DisCarte usa traceroute e sondas com a opção
\emph{Record Route} do protocolo IP ativada para coletar duas sequências
relacionadas de roteadores em caminhos da
Internet~\cite{sherwood08discarte}.  O DisCarte pós-processa essas
sequências com ferramentas de aprendizado de máquina para combiná-las em
uma topologia mais precisa.  Estas e outras técnicas enviam sondas
adicionais e aumentam o custo de mapeamento da topologia.  O objetivo do
RemapRoute é complementar: reduzir o custo do remapeamento de mudanças
de roteamento, aumentando a disponibilidade de sondas para coleta de
topologias mais precisas.  

% Várias técnicas coletam informações para converter topologias de rede
% no nível de endereços IP no nível de roteadores~\cite{sherry10alias},
% pontos de presença~\cite{feldman07popmap} e sistemas
% autônomos~\cite{chen09sidewalk}.  

Nosso trabalho é mais relacionado com técnicas para aumentar a
frequência de medições da topologia da Internet.  Em geral, monitores
têm banda de rede limitada para mapear a topologia.  Reduzir o custo de
cada medição da topologia aumenta diretamente a frequência com a qual
medições podem ser coletadas.  O RocketFuel, por exemplo, reduz o custo
para mapear a topologia de um sistema autônomo alvo descartando caminhos
que têm roteadores de ingresso e egresso no sistema autônomo alvo
idênticos a outro caminho já medido~\cite{spring02rocketfuel}.  Outra
abordagem é reduzir o custo de medições escolhendo apenas um destino em
cada sub-rede num sistema autônomo~\cite{beverly10hifreq}.  O Doubletree
reduz sondas redundantes nos roteadores próximos a monitores
(compartilhados pelos caminhos partindo do monitor) e nos roteadores
próximos a destinos (compartilhados pelos caminhos que terminam no
destino)~\cite{donnet05topology}.  O \rmprt{} foi desenvolvido para
reduzir o custo do remapeamento de mudanças no \dtrack{}, nosso sistema
de mapeamento topológico~\cite{cunha11dtrack}.  O \rmprt{} e o \dtrack{}
complementam as técnicas existentes para redução do custo de mapeamento
topológico.  O \dtrack{} reduz sondas redundantes para roteadores
próximos aos monitores como o Doubletree e é compatível com técnicas
descritas acima para selecionar quais caminhos mapear.
