\section{Definições e fundamentos}
\label{sec:background}

Seguindo a nomenclatura proposta por Paxson \cite{paxson97routing},
chamamos de \emph{caminho virtual} a conectividade entre uma origem e um
destino na Internet.  Em um dado momento, um caminho virtual é
instanciado por uma \emph{rota}.  Devido a mudanças de roteamento, um
caminho pode ser visto como um processo contínuo $C(t)$ que muda de uma
rota a outra ao longo do tempo.  Uma rota é composta de \emph{saltos}
(\emph{hops}) que são instanciados por roteadores.  Saltos são
enumerados a partir da origem e nos referimos a um salto $s$ numa rota
$C(t)$ por $C(t)[s]$.  Uma rota pode ser \emph{simples} se ela tem
apenas uma sequência de roteadores da origem ao destino; ou
\emph{ramificada} se ela tem roteadores que realizam balanceamento de
carga e múltiplas sequências sobrepostas de roteadores da origem ao
destino.  Todos os saltos numa rota simples têm apenas um roteador e
pelo menos um salto numa rota ramificada tem mais de um roteador.

Dadas duas medições consecutivas de um caminho nos instantes $C(t_i)$ e
$C(t_{i+1})$, definimos uma \emph{mudança de caminho} como uma sequência
de saltos contíguos no novo caminho que altera o caminho antigo.
Computamos mudanças de caminho minimizando o número de edições (adição,
remoção e substituição de saltos) necessárias para transformar a nova
rota na rota antiga.  Definimos o \emph{salto de divergência} $s_d$ e o
\emph{salto de convergência} $s_c$ de uma mudança como os saltos
imediatamente anterior e posterior aos saltos editados pela mudança,
respectivamente.  Dizemos que o salto de divergência, o salto de
convergência e todos os saltos entre eles estão \emph{envolvidos} na
mudança de roteamento.  Exemplificando, se $C(t_i) = \{a, b, c, d, e, f,
g\}$ e $C(t_{i+1}) = \{a, b, e, x, y, g\}$, temos uma mudança com $s_d =
1$ e $s_c = 2$ (remoção de $c$ e $d$), e outra mudança com $s_d = 2$ e
$s_c = 5$ (troca de $f$ por $x$ e $y$).
